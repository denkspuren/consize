\chapter{Funktionale Programmierung}
%\selectlanguage{ngerman}

Consize ist eine funktionale Sprache. Also müssen Funktionen -- soviel lässt die Klassifizierung erraten -- eine bedeutende Rolle spielen. Zwar wissen Sie, dass primitive Wörter wie \verb|dup| oder \verb|push| im Wörterbuch mit Funktionen assoziiert sind. Ansonsten geht es aber immer nur um Wörter, Stapel, Mappings und bisweilen um "`Nichts"'. Die Frage ist durchaus berechtigt: Was hat das denn nun alles mit Funktionen zu tun? Diese Frage möchte Ihnen dieses Kapitel beantworten.

Eine zweite Frage schließt sich an: Wie denkt es sich funktional? Wenn Sie von einer imperativen Programmiersprache kommen, werden Sie einige Dinge vermissen und andere merkwürdig finden. Dieses Kapitel will Ihnen einen Einstieg in die funktionale Denke mit Consize bieten.

\section{Die Idee der Funktion} 

In einer funktionalen Programmiersprache steht das Konzept der Funktion im Mittelpunkt. Und in der Tat ist damit der Funktionsbegriff aus der Mathematik gemeint.

Im Mathematik-Unterricht haben Sie Funktionen z.B. geschrieben als $f(x)=x^2+2x$. Man hat Ihnen erklärt, dass die Variable $x$ einen beliebigen Wert aus einer Menge repräsentiert, wie z.B. aus der Menge der Natürlichen Zahlen $\mathbb{N}$, und dass die Ergebniswerte ebenfalls Elemente einer Menge sind. Sie haben das notiert als $f:\mathbb{N}\rightarrow\mathbb{N}$. Eine Funktion bildet Werte aus einer Ausgangsmenge auf Werte einer Zielmenge ab. Zur Berechnung konkreter Funktionswerte haben Sie dann so etwas geschrieben wie $f(3)=15$.

In der Schule geht man in der Regel ein wenig darüber hinweg, sauber zu unterscheiden zwischen der Definition einer Funktion und der Anwendung einer Funktion.

\begin{itemize}
\item Eine {\bf Funktionsdefinition} wie $f(x)=x^2+2x$ legt fest, dass es eine Funktion unter dem Namen $f$ gibt, die ein Argument namens $x$ erwartet und die gleich dem Ausdruck $x^2+2x$ ist. Gerade der letzte Punkt ist wichtig: Die Funktion dient als Abstraktion eines Rechenausdrucks, in dem Fall eines arithmetischen Ausdrucks. Oder anders herum: Der arithmetische Ausdruck löst die Funktion auf. Die mathematische Schreibung mit dem Gleichheitszeichen zieht keine Interpretation der anderen vor.
\item Eine {\bf Funktionsanwendung} wie $f(3)$ wendet die Funktion $f$ auf den Wert $3$ an. Das heißt: Der Wert $3$ ersetzt $x$ (der Fachausdruck ist "`substituiert"'), was zur Folge hat, dass gleichermaßen das $x$ im Rechenausdruck durch $3$ substituiert wird. Damit lässt sich der Rechenausdruck nach den arithmetischen Regeln zu $15$ auswerten -- das Ergebnis der Funktionsanwendung. Sprich $f(9)=15$. Der Wert $f(9)$ kann durch $15$ ersetzt werden.
\end{itemize}

Das Problem mit der Funktionsdefinition ist, dass sie nicht sauber unterscheidet zwischen dem Namen einer Funktion und der Idee der Funktion, die als Abstraktion für einen Rechenausdruck dient. Dieser Problematik nahm sich \href{http://en.wikipedia.org/wiki/Alonzo_Church}{\sc Alonzo Church} an. Er erfand den sogenannten \href{http://de.wikipedia.org/wiki/Lambda-Kalk%C3%BCl}{Lambda-Kalkül}.
Eine namensfreie, anonyme Funktion wird mit dem griechischen Symbol $\lambda$ (ein kleines "`Lambda"') ausgewiesen. Die Schreibweise $f(x)=x^2+2x$ ist vielmehr eine Kurzform für

$$f = \lambda(x)(x^2+2x)$$

Da Mathematiker gerne auf Klammern verzichten, finden Sie meist die etwas veränderte Notation

$$f = \lambda x.x^2+2x$$

Das meint: $f$ ist eine Variable -- nicht anders als es $x$ ist --, die gleichgesetzt ist mit einer anonymen Funktion, die eine Variable $x$ als Argument hat. Und diese Funktion abstrahiert den Rechenausdruck $x^2+2x$.

Die Funktionsanwendung $f(3)$ (manchmal auch ohne Klammern als $f\; 3$ geschrieben) übersetzt sich zu

$$f(3) = (\lambda x.x^2+2x)(3) = 3^2+2\cdot 3 = 15$$

Der Lambda-Kalkül kennt übrigens nur das Konzept der Funktion.\footnote{Es heißt tatsächlich "`der Lambda-Kalkül"' und nicht "`das Lambda-Kalkül"'.} Es gibt keine booleschen Werte für \emph{wahr} und \emph{falsch}, keine Zahlen, keine Stapel, keine Mappings, nichts dergleichen. Der Witz ist, dass Daten mittels Funktionen kodiert werden. Das scheint auf den ersten Blick erstaunlich zu sein, ist es aber nicht wirklich. Ihr Rechner kodiert alles mit Einsen und Nullen. Die Bitfolge \verb|10110111| mag ebenso eine Instruktion wie auch ein Datum sein. Ähnlich kann man auch alles mit Funktionen kodieren.

Für praktische Zwecke verkompliziert die Beschränkung auf Funktionen unsere Betrachtungen. Gehen wir also davon aus, dass es neben den Funktionen noch Daten gibt, und zwar genau die, die auch in Consize bekannt sind: Wörter, Stapel, Mappings und \verb|nil|. Für die Menge aller Wörter verwenden wir den Großbuchstaben $W$, für die Menge aller Stapel $S$, für Mappings $M$, für die Menge mit dem Element \verb|nil| $N$. Ein beliebiges Datum (ein \verb|item|) kann ein Element aus der Vereinigungsmenge $I$ sein; Funktionen lassen wir hier absichtlich als Daten außen vor.

\begin{equation}
I = W\cup S \cup M \cup N
\end{equation}

Und noch eine Annahme machen wir, damit der Bezug zu Consize einfach bleibt: Eine Funktion kann grundsätzlich nur einen Stapel als Argument entgegen nehmen; und sie liefert immer einen Stapel als Ergebnis zurück. An diese Regel halten sich ausnahmslos alle Funktionen der Consize-VM. Formal ausgedrückt heißt das: Jede beliebige Funktion in Consize bildet einen Stapel auf einen Stapel ab: $S\rightarrow S$. Den Fehlerfall, dass eine Funktion einen ungültigen Stapel zur Verarbeitung vorgelegt bekommt, lassen wir der Einfachkeit halber unberücksichtigt. Die Menge aller Funktionen, die einen Stapel verarbeiten und einen Stapel zurückgeben, bezeichnen wir mit $F$.

\section{Funktionskomposition}

Mit der Consize-VM ist Ihnen ein fest vorgegebener Satz an stapelverarbeitenden Funktionen gegeben. Damit ließe sich nicht viel anfangen, wenn es die Funktionkomposition nicht gäbe. Aus zwei oder mehr Funktionen kann eine neue Funktion erstellt werden.

Die Komposition zweier Funktionen $f$ und $g$, geschrieben als $f\circ g$ ist definiert als

\begin{equation}
(f\circ g)(s) = g(f(s))
\end{equation}

Die Idee der Funktionskomposition ergibt sich aus der verschachtelten Funktionsanwendung. Erst wird $f$ auf $s$ angewendet, dann $g$ auf das sich ergebende Resultat von $f(s)$. Damit die Komposition funktioniert, muss das Resultat $f(s)$ vom Typ her zum Typ des Arguments von $g$ passen. Da hier nur Stapel umhergereicht werden, ist die Voraussetzung erfüllt.

% Die Identitätsfunktion $id(s)=s$ ist das neutrale Element der Funktionskomposition. Es gilt für eine beliebige Funktion f

Mit Hilfe der Funktionskomposition lassen sich neue Funktionen definieren, wie z.B. die Funktion $-rot$ aus der doppelten Anwendung der Funktion $rot$:

\begin{equation}
-rot = rot \circ rot
\end{equation}


% [ -rot ] = [ rot rot ]

Eine Funktion wie $unpush$ läßt sich definieren als

\begin{equation}
unpush = dup \circ pop \circ swap \circ top
\end{equation}

Wie Sie wissen, arbeiten Sie in Consize nicht direkt mit Funktionen, sondern mit Wörtern, genauer, mit Wörtern in Stapeln, sogenannten Quotierungen. Lediglich die primitiven Wörter sind über das Wörterbuch mit Funktionen assoziiert. Nicht-primitive Wörter können (müssen aber nicht) mit einer Quotierung assoziiert sein.

Es ist der Interpreter der Consize-VM, d.h.\ die über \verb|stepcc| gegebene Funktion, die einen Bezug herstellt von Quotierungen zu Funktionen und zur Funktionskomposition. Die dahinter stehende Mathematik basiert auf drei Formeln. Die ersten beiden Formeln sind die Basis für einen sogenannten Homomorphismus: Funktionen und Funktionskomposition finden ihre Entsprechung in Quotierungen und der Konkatenation. Die dritte Formel definiert den Umgang mit Daten.

\section{Die Mathematik zu Consize}

Die Mathematik zu Consize benötigt nicht unbedingt die Idee eines Stapels, sondern -- etwas freier -- die Idee einer Folge (\emph{sequence}) von null oder mehr Elementen $[s_1,\dots,s_n]$. Die Konkatenation $\oplus$ zweier Stapel bzw. Folgen $s$ und $s'$ ergibt sich aus der Zusammenstellung der Elemente beider Folgen zu einer neuen Folge unter Beibehaltung der Reihenfolge der Elemente:

\begin{equation}
s\oplus s' = [s_1,\dots,s_n]\oplus [s'_1,\dots,s'_m]
           = [s_1,\dots,s_n,s'_1,\dots,s'_m]
\end{equation}


Der leere Stapel bzw. die leere Folge ist das neutrale Element der Konkatenation. Es gilt für einen beliebigen Stapel $s$:

\begin{equation}
s\oplus[\;] = [\;]\oplus s = s
\end{equation}


Programme in Consize sind immer Quotierungen, also Stapel mit irgendwelchen Elementen aus $I$. Als Abstraktion gilt der Mechanismus, eine Quotierung mit einem bislang nicht verwendetem Wort mit einer beliebigen Quotierung gleichzusetzen. Zum Beispiel definiert sich das Wort \verb|-rot| zu

\verb|[ -rot ] = [ rot rot ]|

Sie erahnen sicherlich die Nähe zur oben dargestellten Funktionskomposition. Bevor wir darauf im Detail eingehen, wollen wir das Wörterbuch der Consize-VM explizit mit Hilfe einer Funktion $vm: [ W ] \rightarrow F$ abbilden. Nur Quotierungen mit einem einzigen Wort können mit einer Funktion assoziiert sein.\footnote{Das Wörterbuch der Consize-VM hat als Schlüsseleinträge Wörter und keine in einem Stapel verpackte Wörter. Der Grund ist, dass \texttt{stepcc} das Wort bereits aus einer Quotierung extrahiert hat.}

\begin{description}
\item[Aufruf einer Quotierung] Der Aufruf eines Stapels $s$ mit einer Quotierung $q$ als oberstem Element ($[q]\oplus s$)\footnote{Achtung, $q\oplus s$ konkateniert Quotierung $q$ und Sequenz $s$. Es soll aber die Quotierung als Element ganz links in der Sequenz auftauchen! Darum muss es $[q]\oplus s$ heißen.} ist gleich der mit der Quotierung assozierten Funktion angewendet auf den Stapel $s$.

\begin{equation}
call([q]\oplus s) = vm(q)(s)
\end{equation}

Der Aufruf einer Quotierung definiert die Semantik (Bedeutung, Interpretation) eines Stapels als Programm.

\item[Dekomposition einer Quotierung] Eine aus mehreren Elementen bestehende Quotierung $q$ kann als Konkatenation von Teilquotierungen $q=q_L\oplus q_R$ verstanden werden. Die Interpretation der Konkatenation übersetzt sich in die Funktionskomposition der Interpretationen der Teilquotierungen.

\begin{equation}
vm(q_L\oplus q_R) = vm(q_L) \circ vm(q_R)
\end{equation}

Faktisch zerlegt diese Gleichung eine zu interpretierende Quotierung in ihre kleinstmöglichen Teilprogramme, in eine Reihe von Quotierungen mit nur einem Element. Zu diesen einelementigen Quotierungen müssen über $vm$ jeweils Funktionen abgebildet sein. Eine leere Quotierung findet ihre funktionale Entsprechung in der Identitätsfunktion $id(s)=s$ -- die Sequenz $s$ bleibt unverändert. Wenn kein Programmierfehler vorliegt, lässt sich jede Quotierung auf Daten und primitive Wörter der Virtuellen Maschine herunterbrechen.
\end{description}
% $$call([[i]\oplus r]\oplus s) = (vm([i]) \ circ 

% call([[i](+)r](+)s) = vm([i])

Apropos Daten. Es muss eine Möglichkeit geben, Daten (Stapel, Mappings, "Nichts" und Wörter, die als Daten verstanden und nicht interpretiert werden sollen) in einer Quotierung unangetastet auf dem Stapel abzulegen. Dazu dient die Funktion $self(i)$.

\begin{description}
\item[Interpretation von Daten] Die Funktion $self(i)$ nimmt ein Element $i$ (\emph{item}) entgegen und erzeugt eine Funktion, die eine Sequenz $s$ als Argument erwartet, um das Element $i$ der Sequenz hinzuzufügen.\footnote{Die Gleichung ist eine Kurzform von $self = \lambda i.(\lambda s.[i]\oplus s)$.}

\begin{equation}
self(i) = \lambda s.[i]\oplus s
\end{equation}

Somit gilt für die Interpretation eines Datenelements $i$ in einer Quotierung:

\begin{equation}
vm([i]) = self(i)
\end{equation}

\end{description}

Was wollen uns die Formeln eigentlich sagen? Im Grunde ist die Angelegenheit glasklar, aber man muss sich an diese sehr kondensierte Ausdruckform der Mathematik gewöhnen. Freisprachlich ausgedrückt definieren die Formeln die Bedeutung einer Quotierung als Programm. Eine Quotierung, verstanden als eine Konkatenation von Teilquotierungen mit nur einem Element, hat eine klare funktionale Entsprechung, wenn man sie aufruft: Die Auflösung der Teilquotierung findet eine Abbildung auf Funktionen der VM, die Konkatenation wird zur Funktionskomposition. Das ist es, was Consize zu einer funktionalen Sprache macht.

Mit den Worten der Mathematik ausgedrückt: Die Abbildung des Monoids $(S,\oplus,[\;])$ auf den Monoiden $(F,\circ,id)$ ist ein Homomorphismus.

Jetzt, da Ihnen die Funktion $self$ bekannt ist, kann die Funktions

\begin{equation}
call([q]\oplus s) = vm(q)(s)
\end{equation}

mit $q$ als Quotierung und $s$ als Sequenz auch ausgedrückt werden als

\begin{equation}
self(q)\circ call = vm(q)
\end{equation}

Die Anwendung von $vm(q)$ auf $s$ tritt dabei vollkommen in den Hintergrund.\footnote{$(self(q)\circ call)(s)=call([q]\oplus s)=vm(q)(s)$} Das "`Befüllen"' der Sequenz $s$ kann vollständig über $self$ erfolgen und ist damit Teil eines konkatenativen Programms. Damit reicht es, zur Auswertung $vm(q)$ lediglich auf eine leere Sequenz anzuwenden.

\begin{align}
self(q)\circ vm([\mathtt{call}]) &= vm(q) \\
    vm([q]\oplus[\mathtt{call}]) &= vm(q) 
\end{align}


Der Homomorphismus, der eine Quotierung auf die Komposition von Funktionen abbildet, macht allein einen Formalismus noch nicht turingvollständig. Es fehlt der grundlegende Mechanismus des Selbstbezugs im Sinne eines Selbstaufrufs. Das ist leicht zu realisieren, indem \verb|call| Teil der Virtuellen Maschine wird.

\begin{equation}
vm([\mathtt{call}])=call % Selbstbezug!!!
\end{equation}

Mit dieser Definition untersuchen wir das Verhalten von \verb|call| als Bestandteil einer Quotierung. Mit \verb|call| sei eine Quotierung $q$ aufgerufen; $r$ bilde den Rest der Quotierung ab.

\begin{align}
self([q]\oplus [\mathtt{call}]\oplus r)\circ call
&= vm([q]\oplus    [\mathtt{call}]  \oplus r)    \\
&= vm([q])\circ vm([\mathtt{call}]) \circ vm(r)  \\
&=  vm([q])\circ call                \circ vm(r) \\
&= self(q) \circ call                \circ vm(r) \\
&=   vm(q)                           \circ vm(r) \\
&=   vm(q         \oplus                      r) \\
&= self(q\oplus r) \circ call
\end{align}

Die Gleichungen fördern eine interessante Eigenschaft ans Tageslicht. Der Aufruf einer Quotierung per \verb|call| entspricht einer Konkatenation der Quotierung mit dem Quotierungsanteil nach \verb|call|. Statt den \verb|call| einer Quotierung funktional über die Funktion $call$ aufzulösen, kann alternativ die Konkatenation der Quotierung mit dem Programmanteil nach \verb|call| durchgeführt werden. Es ist unnötig, $call$ als Funktion bereitzustellen, sofern die Virtuelle Maschine mit dem Auftreten des Wortes \verb|call| umzugehen weiß. Unter der Voraussetzung, dass sich auf dem anzuwendenden Stapel $s$ eine Quotierung $q$ befindet, gilt nämlich:

\begin{equation}
vm([q]\oplus[\mathtt{call}]\oplus r) =
vm(q \oplus r)
\end{equation}

Der Consize-Interpreter arbeitet genau mit dieser Umsetzung des Wortes \verb|call|.\footnote{Mit der Variante $vm([\mathtt{call}]\oplus r)([q]\oplus s)=vm(q\oplus r)(s)$.}


% Damit: Ableitung, dass Interpreter auf konkatenativer Ebene möglich ist!

% call(q) = vm(q)
% vm(q1(+)q2) = vm(q1) o vm(q2)

% => call([i](+)r) = vm([i]) o call(r)

% self(i) = \lambda s.[i](+)s

% call([q](+)s) = VM(q)(s)
% VM([l](+)[r]) = VM([l]) o VM([r])
% 
% call([[i](+)r](+)s) =
% VM(q)(s) =
% ( VM([i]) o VM(r) )(s)
% VM(r)(VM([i])(s))






% FUNKTIONALE DENKE
% denke in destruktoren und konstruktoren only
% simuliere niemals einen Index
% verwende niemals size als Kriterium für Rekursionsabbruch.

\section{Funktionen \& Rekursion}
% Die Idee der Rekursion
% benamte Rekursion vs. anonyme Rekursion, Fixpunkt-Kombinator (Y)

\section{filter, map und Co}

\section{Keine Variablen: Stapelschieberei und Kombinatoren}

\section{Seiteneffekte}

\section{Was kümmert mich die Zukunft, wenn ich keine Vergangenheit habe}
% Continuations, Delimited Continuations, Monads

\section{Und sonst?}
% Lazyness

% Die Interaktion mit Consize ist nicht funktional!
% Inkrementeller Aufbau einer Umgebung.
% Für eine Momentaufnahme: funktional

% Engineering: Vor- und Nachbedingungen




\section*{History}
\begin{itemize}
\item[2013-05-18] Start of writing on this chapter
\end{itemize}

